\documentclass[12pt,a4paper]{article}
\usepackage[T1]{fontenc}
\usepackage[utf8]{inputenc}
\usepackage[brazilian]{babel}
\usepackage{amsmath}
\usepackage{amsfonts}
\usepackage{amssymb}
\usepackage{graphicx}

\usepackage{tikz}
\usetikzlibrary{arrows.meta, automata, graphs, graphdrawing, quotes}
\usegdlibrary{force}

\title{Respostas - Autômatos, \texttt{[nfa-repr]}}
\author{Davi Antônio da Silva Santos}
\begin{document}

\section{Questão Q1}
\subsection{Item 1}
O autômato é classificado como não determinístico (NFA) quando possui múltiplas transições possíveis para um mesmo conjunto de entradas em um mesmo estado, ou quando possui múltiplos estados iniciais.

Por exemplo, no autômato em questão, o estado A pode prosseguir para os estados B ou C com um mesmo conjunto de entradas (dígitos de zero a nove).

\subsection{Item 2}
\begin{enumerate}
	\item[a] A entrada \texttt{42} percorre os estados A e B e é aceita
	\item[b] A entrada \texttt{3.14} percorre os estados C, D, E e E e é aceita.
	\item[c] A entrada \texttt{123.} percorre os estados A, C, C e D e não é aceita.
\end{enumerate}

\subsection{Item 3}
Através da conversão do autômato NFA para DFA (tabela \ref{tab:q1_3_dfa} e figura \ref{fig:q1_3_dfa}) usando a tabela \ref{tab:q1_3_nfa}, é possível unificar os estados B e C.

\begin{table}[htb]
\caption[Tabela do autômato original (NFA)]{Tabela do autômato original (NFA).}
\centering
\label{tab:q1_3_nfa}
\begin{tabular}{c|c|c}
    \hline
    & \textbf{0-9} & \textbf{.} \\
    A (inicial) & B, C & \\
    B (final) & B & \\
    C & C & D \\
    D & E & \\
    E (final) & E & \\
    \hline
\end{tabular}
\end{table}

\begin{table}[htb]
\caption[Tabela do autômato convertido (DFA)]{Tabela do autômato convertido (DFA).}
\centering
\label{tab:q1_3_dfa}
\begin{tabular}{c|c|c}
    \hline
    & \textbf{0-9} & \textbf{.} \\
    A (inicial) & BC & \\
    BC (final) & BC & D \\
    D & E & \\
    E (final) & E & \\
    \hline
\end{tabular}
\end{table}

\begin{figure}[htb]
	\centering
	\begin{tikzpicture}[>=stealth,shorten >=1pt,auto,node distance=2cm]
	\node[initial,state]   (A)               {$A$};
	\node[state,accepting] (BC)[right of=A]  {$BC$};
	\node[state]           (D) [right of=BC] {$D$};
	\node[state,accepting] (E) [right of=D]  {$E$};

	\path[->] (A) edge node {0-9} (BC)
		(BC) edge [loop above] node{0-9} (BC)
		(BC) edge node {.} (D)
		(D) edge node {0-9} (E)
		(E) edge [loop above] node {0-9} (E);
	\end{tikzpicture}
	\caption{Diagrama do DFA.}
	\label{fig:q1_3_dfa}
\end{figure}

\section{Questão Q2}
\subsection{Item 1}
Sequências de $n$ a's, onde $n$ é um múltiplo de 3 ou 5. Ex.: $a{12}$, onde $12 = 3 \cdot 4$. Esse autômato está descrito na tabela \ref{tab:q2_1_nfa} e na figura \ref{fig:q2_1_dfa}.
\begin{table}[htb]
\caption[Tabela do autômato (NFA)]{Tabela do autômato (NFA).}
\centering
\label{tab:q2_1_nfa}
\begin{tabular}{c|c}
    \hline
    & \textbf{a} \\
    A (inicial) & B, E \\
    B & C \\
    C & D\\
    D (final) & E \\
    E & F \\
    F & G \\
    G & H \\
    H & I \\
    I (final) & E \\
    \hline
\end{tabular}
\end{table}

\begin{figure}[htb]
	\centering
	\begin{tikzpicture}[>=stealth,shorten >=1pt,auto,node distance=1.1cm, spring electrical layout, electric charge=25, electric force order=4]
	\node[initial,state]   (A)  {$A$};
	\node[state]           (B)  {$B$};
	\node[state]           (C)  {$C$};
	\node[state,accepting] (D)  {$D$};
	\node[state]           (E)  {$E$};
	\node[state]           (F)  {$F$};
	\node[state]           (G)  {$G$};
	\node[state]           (H)  {$H$};
	\node[state,accepting] (I)  {$I$};

	\graph[edge label=a]{
		(A) -> (B),
		(A) -> (E),
		(B) -> (C),
		(C) -> (D),
		(D) -> (B),
		(E) -> (F),
		(F) -> (G),
		(G) -> (H),
		(H) -> (I),
		(I) -> (E)
	};

	\end{tikzpicture}
	\caption{Diagrama do DFA.}
	\label{fig:q2_1_dfa}
\end{figure}

\end{document}
