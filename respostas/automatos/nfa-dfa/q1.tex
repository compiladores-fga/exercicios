\documentclass[12pt,a4paper]{article}
\usepackage[T1]{fontenc}
\usepackage[utf8]{inputenc}
\usepackage[brazilian]{babel}
\usepackage{amsmath}
\usepackage{amsfonts}
\usepackage{amssymb}
\usepackage{graphicx}

\usepackage{tikz}
\usetikzlibrary{arrows.meta, automata, graphs, graphdrawing, quotes}
\usegdlibrary{force, layered}

\usepackage[counterclockwise]{rotating}
\usepackage[unicode]{hyperref}
\usepackage{placeins}

\title{Respostas - Autômatos, \texttt{[nfa-dfa]}}
\author{Davi Antônio da Silva Santos}
\begin{document}

\section{Questão Q1}
\subsection{Item a}
Figura \ref{fig:q1_a_nfa}

\begin{figure}[htb]
	\centering
	\begin{tikzpicture}[>=stealth,shorten >=1pt,auto, layered layout, grow'=right, level distance=2cm, sibling distance=2cm]
	\node[state,initial,accepting] (ABC)  {$ABC$};
	\node[state] (D)  {$D$};
	\node[state] (E)  {$E$};
	\node[state,accepting] (C)  {$C$};

	\graph{
		(ABC) ->[edge label=a]          (D),
		(ABC) ->[edge label=b]          (E),
		(D) ->[edge label=b]          (E),
		(E) ->[edge label=a]          (C)
	};

	\end{tikzpicture}
	\caption{Diagrama do DFA convertido a partir do NFA-$\epsilon$.}
	\label{fig:q1_a_nfa}
\end{figure}
\FloatBarrier

\subsection{Item b}
Figura \ref{fig:q1_b_nfa}

\begin{figure}[htb]
	\centering
	\begin{tikzpicture}[>=stealth,shorten >=1pt,auto, layered layout, grow=right, level distance=2cm]
	\node[state,initial,accepting] (ABD)  {$ABD$};
	\node[state,accepting] (C)  {$C$};
	\node[draw=none] (dummy) {};

	\graph{
		(ABD) ->[edge label=a]          (C),
		(C) ->[edge label=a, loop right]          (dummy),
	};

	\end{tikzpicture}
	\caption{Diagrama do DFA convertido a partir do NFA-$\epsilon$.}
	\label{fig:q1_b_nfa}
\end{figure}

\end{document}