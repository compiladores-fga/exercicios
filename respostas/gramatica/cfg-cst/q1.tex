\documentclass[12pt,a4paper]{article}
\usepackage[T1]{fontenc}
\usepackage[utf8]{inputenc}
\usepackage[brazilian]{babel}
\usepackage{amsmath}
\usepackage{amsfonts}
\usepackage{amssymb}
\usepackage{graphicx}

\usepackage{tikz}
\usetikzlibrary{arrows.meta, automata, graphs, graphdrawing, quotes}
\usegdlibrary{force, layered, trees}

\usepackage[counterclockwise]{rotating}
\usepackage[unicode]{hyperref}
\usepackage{placeins}

\title{Respostas - Gramática, \texttt{[cfg-cst]}}
\author{Davi Antônio da Silva Santos}
\begin{document}

\section{Questão Q1}
\subsection{Item 1}
\begin{figure}[htb]
	\centering
	\begin{tikzpicture}[>=stealth,shorten >=1pt,auto, tree layout]

	\graph[nodes={circle, draw}, minimum number of children=2]{
		"expr" -> "term" -> "atom" -> "NUMBER" -> "42"
	};

	\end{tikzpicture}
	\caption{Árvore sintática concreta da entrada \texttt{42}.}
\end{figure}
\FloatBarrier

\subsection{Item 2}
\begin{figure}[htb]
	\centering
	\begin{tikzpicture}[>=stealth,shorten >=1pt,auto, tree layout]

	\graph[nodes={circle, draw}, minimum number of children=2]{
		"expr" ->
		{"term" -> "atom" -> "NUMBER" -> "2",
		"*",
		"term1"[as=term] ->
		{"(", "expr1"[as=expr] ->
		{"term2"[as=term] -> "atom1"[as=atom] -> "NUMBER1"[as=NUMBER] -> 11, "+", "term3"[as=term] ->
		{"atom2"[as=atom] -> "NUMBER2"[as=NUMBER] -> "21"[as=2], "*2"[as=*], "atom3"[as=atom] -> "NUMBER3"[as=NUMBER] -> "5"}}, ")"}}
	};

	\end{tikzpicture}
	\caption{Árvore sintática concreta da entrada \texttt{2*(11+2*5)}.}
\end{figure}

\end{document}
