\documentclass[12pt,a4paper]{article}
\usepackage[T1]{fontenc}
\usepackage[utf8]{inputenc}
\usepackage[brazilian]{babel}
\usepackage{amsmath}
\usepackage{amsfonts}
\usepackage{amssymb}
\usepackage{graphicx}

\usepackage{tikz}
\usetikzlibrary{arrows.meta, automata, graphs, graphdrawing, quotes}
\usegdlibrary{force, layered}

\usepackage[counterclockwise]{rotating}
\usepackage[unicode]{hyperref}
\usepackage{placeins}

\title{Respostas - Autômatos, \texttt{[nfa-epsilon]}}
\author{Davi Antônio da Silva Santos}
\begin{document}

\section{Questão Q1}
\subsection{Item 1}
Um caminho para \texttt{[aa, a]} é a transição entre os estados A, B, C, D, C, D, C, D, E, F.

\subsection{Item 2}
Após a entrada \texttt{[aaa} é possível que o autômato esteja nos estados C, D ou E.

\section{Questão Q2}
\subsection{Item 1}
Figura \ref{fig:q2_1_nfa}

\begin{figure}[htb]
	\centering
	\begin{tikzpicture}[>=stealth,shorten >=1pt,auto, layered layout, grow=right, level distance=2cm, initial where=below]
	\node[initial,state,accepting] (A)  {$A$};
	\node[initial,state,accepting] (B)  {$B$};
	\node[initial,state,accepting] (C)  {$C$};
	\node[initial,state,accepting] (D)  {$D$};

	\graph{
		(A) -> [edge label=a]          (B),
		(A) -> [edge label=b]          (C),
		(A) -> [edge label=c]          (D),
		
		(B) -> [edge label=b]          (C),
		(B) -> [edge label=c,bend right=90,looseness=1.5]          (D),
		
		(C) -> [edge label=c]          (D);
		{[same layer] (A), (B), (C), (D)};
	};

	\end{tikzpicture}
	\caption{Diagrama do NFA sem transições $\epsilon$.}
	\label{fig:q2_1_nfa}
\end{figure}
\FloatBarrier

\subsection{Item 1}
Figura \ref{fig:q2_2_nfa}

\begin{figure}[htb]
	\centering
	\begin{tikzpicture}[>=stealth,shorten >=1pt,auto, layered layout, grow=right, level distance=3.5cm]
	\node[initial,state] (A)  {$A$};
	\node[state] (B)  {$B$};
	\node[state] (C)  {$C$};
	\node[state,accepting] (D)  {$D$};
	\node[draw=none] (dummy) {};
	
	\graph{
		(A) -> [edge label=a]          (B),
		
		(B) ->[edge label=a,loop below] (dummy),
		(B) -> [edge label=b]          (C),
		
		(C) -> [edge label=a, bend right=60]          (B),
		(C) -> [edge label=b,loop below]          (dummy),
		(C) -> [edge label=c]          (D),
		
		(D) -> [edge label=a, bend right=60]          (B),
		(D) -> [edge label=b, bend right=60]          (C),
		(D) -> [edge label=c,loop below]          (dummy),		
	};

	\end{tikzpicture}
	\caption{Diagrama do NFA sem transições $\epsilon$.}
	\label{fig:q2_2_nfa}
\end{figure}

\end{document}
